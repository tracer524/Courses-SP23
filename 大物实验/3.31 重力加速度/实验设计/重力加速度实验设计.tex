\documentclass[UTF8]{ctexart}
\usepackage[a4paper,left=3cm,right=3cm,top=2cm]{geometry}

\setlength\lineskiplimit{5.25bp}
\setlength\lineskip{5.25bp}

\title{重力加速度的测量——实验设计}
\author{崔士强}
\date{\today}

\bibliographystyle{plain}

\begin{document}

\maketitle
\section{不确定度的计算}
\noindent 由周期公式
\[T=2\pi \sqrt{\frac{l}{g}}\]
得最大不确定度公式
\[\frac{\Delta g}{g}=2\frac{\Delta T}{T}+\frac{\Delta l}{l}\]
实验要求 $\Delta g/g<1\%$,由不确定度均分原理可得 $\Delta T/T<0.25\%$,$\Delta l/l<0.5\%$

\noindent 设 $\Delta_1$ 为秒表的最大允差 $0.01s$ ,$\Delta_2$ 为实验人员测量时间的精度 $0.2s$ ,则有
\[\Delta T=\Delta_{1}+\Delta_{2}\]
可得总时间的最小值
\[T_{min}=84s\]
取 $\Delta l=0.2cm$ 可得摆长的最小值
\[l_{min}=40cm\]
\section{实验过程}
使用钢卷尺直接测量细绳悬挂点到小球中心的距离作为摆长,受细绳自重的影响,摆长过长会导致测量精度下降.
此实验中摆长 $l$ 取 $70cm$.

取合肥地区重力加速度的估计值 $g=9.7947m/s^2$ ,代入周期公式得 $T\approx 1.68s$ ,继而得出周期数的最小值
\[n_{min}=50\]
\indent 由于摆长直接测出,故不需要再对小球直径进行测量.
实验过程中为了减小误差,从球第一次经过最低点开始计时,50个周期后停止计时得到总时间,重复进行五次测量,计算摆动周期便可求出重力加速度.
\bibliography{math}

\end{document}